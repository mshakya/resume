%%%%%%%%%%%%%%%%%%%%%%%%%%%%%%%%%%%%%%%%%
% Medium Length Professional CV
% LaTeX Template
% Version 2.0 (8/5/13)
%
% This template has been downloaded from:
% http://www.LaTeXTemplates.com
%
% Original author:
% Trey Hunner (http://www.treyhunner.com/)
%
% Important note:
% This template requires the resume.cls file to be in the same directory as the
% .tex file. The resume.cls file provides the resume style used for structuring the
% document.
%
%%%%%%%%%%%%%%%%%%%%%%%%%%%%%%%%%%%%%%%%%

%----------------------------------------------------------------------------------------
%	PACKAGES AND OTHER DOCUMENT CONFIGURATIONS
%----------------------------------------------------------------------------------------

\documentclass{resume} % Use the custom resume.cls style


\usepackage[left=0.5in,top=0.6in,right=0.55in,bottom=0.6in]{geometry} % Document margins
\name{Migun Shakya, PhD} % Your name
\address{75 College St. \\ Hanover, NH 03755} % Your address
%\address{123 Pleasant Lane \\ City, State 12345} % Your secondary addess (optional)
\address{(603)~$\cdot$~646~$\cdot$~9397 \\ migun.shakya@dartmouth.edu} % Your phone number and email
\begin{document}




%----------------------------------------------------------------------------------------
%	EDUCATION SECTION
%----------------------------------------------------------------------------------------

\begin{rSection}{Education}

{\bf Ph.D} Genome Sciences {$\diamond$} University of Tenneessee {$\diamond$} Knoxville, TN \hfill {2008 - 2013} 
\\
\\
{\bf B.S}  Biology {$\diamond$} Southwerstern Oklahoma State University {$\diamond$}  Weatherford, OK \hfill{2008}
\end{rSection}

%----------------------------------------------------------------------------------------
%	RESEARCH EXPERIENCE SECTION
%----------------------------------------------------------------------------------------

\begin{rSection}{Research Experience}
\begin{rSubsection}{Postdoctoral Associate} {2013 - Present}{}{}
\item Department of Biology, Dartmouth College, Hanover, NH
\item Evolution of Gene Transfer Agents across Alphaproteobacteria
\end{rSubsection}

%------------------------------------------------
\begin{rSubsection}{Graduate Research Assistant} {August 2009 - August 2013}{}{}
\item Genome Science and Technology Program
\item University of Tennessee and Oak Ridge National Lab, Oak Ridge, TN 
\item Dissertation: Validating Approaches for Studying Microbial Diversity to Characterize Communities from Roots of \textit{Populus deltoides}
\end{rSubsection}
\end{rSection}

%------------------------------------------------

\begin{rSection}{Publications}
\item Hurt, R.A., Robeson II M.S., \textbf{Shakya M.}, Moberly, J.G., Vishnivetskaya, T.A., Gu B., and Elias, D.A. Improved Yield of High Molecular Weight DNA Coincides with Increased Microbial Diversity Access from Iron Oxide Cemented Sub-Surface Clay Environments. \textit{Plos ONE} 2014.
 \item \textbf{Shakya, M.}, Gottel N., Castro H., Yang Z.K., Gunter L., Labbe J., Muchero W., Bonito G., Vilgalys R., Tuskan G., Podar M., Schadt C.W. A Multifactor Analysis of Fungal and Bacterial Community Structure in the Root Microbiome of Mature Populus deltoides Trees. \textit{Plos ONE} 2013. \doi{10.1371/journal.pone.0076382}
\item \textbf{Shakya, M.}, Quince, C., Campbell, J., Yang, Z., Schadt, C.W., Podar, M. Comparative metagenomic and rRNA microbial diversity characterization using Archaea and Bacteria synthetic communities \textit{Environmental Microbiology} 2013.\doi{10.1111/1462-2920.12086}
\item Flores E.G.,\textbf{Shakya, M.}, Meneghin J.,Podar, M., Seewald., J.S., Reysenbach A-L. Inter-field Variability in the Microbial Communities of Hydrothermal Vent Deposits from a Back-Arc Basin.\textit{Geobiology} 2012. \doi{10.1111/ j.1472-4669.2012.00325.x}
\end{rSection}


%------------------------------------------------
%Book Chapter
%------------------------------------------------
\begin{rSection}{Book Chapters}
\item Culley A.I., \textbf{Shakya M.}, Lang A.S. (2015) Viral evolution at the limits. In \textit{Microbial Evolution under Extreme Conditions} (pp.209-222).Germany.degruyter.
\end{rSection}

%------------------------------------------------
%Honors and awards
%------------------------------------------------

\begin{rSection}{Honors and Awards}
\href{http://www.utk.edu}{\textbf{University of Tennessee, Knoxville}}
\item 2011 Genomic Sciences Meeting Student Travel Grant

\item \href{http://www.swosu.edu}{\textbf{Southwestern Oklahoma State University}},
\item In-State Tuition Waiver 2003-2008
\item Ottis and Buena Ballard Memorial Scholarship 2007
\item Who is Who Among American Universities 2008
\end{rSection}

%------------------------------------------------
%Scientific Meeting
%------------------------------------------------

\begin{rSection}{Scientific Conferences}

\item \textbf{Shakya M.} and Zhaxybayeva O. Origin and Evolution of gene transfer agents in $\alpha$-proteobacteria. In:\emph{The Burroughs Wellcome Big Data Symposium}, November 4, 2015. Poster

\item \textbf{Shakya M.} and Zhaxybayeva O. Origin and Evolution of gene transfer agents in $\alpha$-proteobacteria. In:\emph{1st Annual Celebration of Biomedical Research at Dartmouth (C-BRaD)}, September 21, 2015. Poster

\item \textbf{Shakya M.} and Zhaxybayeva O. Origin and Evolution of gene transfer agents in $\alpha$-proteobacteria. In:\emph{Mobile Genetic Elements: In Silico, In Vitro, In Vivo}, September 3-5, 2015. Selected talk

\item \textbf{Shakya M.} and Zhaxybayeva O. Origin and Evolution of gene transfer agents in $\alpha$-proteobacteria. In:\emph{Gordon Research Conference: Microbial Population Biology}, July 19-24, 2015. Poster

\item \textbf{Shakya M.} and Zhaxybayeva O. Distinguishing Gene Transfer Agent genes from their phage homologs. In:\emph{Boston Bacterial Meeting 2014, Boston, MA}, June 12-13, 2014. Poster

\item \textbf{Shakya M.}, Drivers of microbial community structure in rhizosphere and endosphere of \textit{Populus deltoides}.In: \emph{American Society for Microbiology KY-TN Branch Meeting }, October 26-27, 2012.Talk

\item  \textbf{Shakya M.}, Gottel,N.R., Castro,H.F.,Yang Z.K.,Kerley M., Podar M.,Doktcyz, M.J., Schadt, C.W. Archaea associated with Populus and its surrounding soil and trees.In: \emph{International Society of Microbial Ecology 14, Copenhagen, Denmark}, August 19-24, 2012.Poster.


\item Schadt, C.W.,\textbf{Shakya M.}, Gottel,N.R., Castro,H.F.,Yang Z.K.,Kerley M., Bonito G., Labbe J., Muchero W.,Vilgalys R., Tuskan G., Podar M.,Doktcyz, M.J. Roles of genotype-by-environment interactions in shaping the root-associated microbiome of Populus. In:\emph{Ecological Society of America 97, Portland, Oregon} August 5-10, 2012.

\item \textbf{Shakya M.}, Gottel,N.R., Castro,H.F.,Yang Z.,Kerley M., Podar M.,Doktcyz, M.J., Schadt, C.W. Plant-Microbe Interface: Dynamics of Bacterial Microbiome of Populus del- toides In: 2012 Department of Energy Genomic Science Awardee Meeting X, February 26-29, 2012.Poster

\item Graham, E.D, \textbf{Shakya M.}, Xu X., Phelps, T.J., Thornton,P.E., Elias D.A,. Char- acterization and Modeling Of Microbial Carbon Metabolism In Thawing Permafrost. In: American Society of Microbiology, San Francisco, CA, June 16 - 19, 2012.Poster.


\item Bonito, G., Schadt, C., Hameed, K., \textbf{Shakya, M.}, Chen, K. H., Tuskan, G., Vilgalys, R. (2012).Rhizospheric mycobiota associated with \textit{Populus deltoides}. In:\emph{Mid-Atlantic States Mycological Conference, University of Tennessee, Knoxville} April 13-15, 2012.

\item \textbf{Shakya M.}, Gottel,N.R., Castro,H.F.,Yang Z.,Kerley M., Podar M.,Doktcyz, M.J.,Schadt, C.W. Plant-Microbe Interface: Dynamics of Bacterial Microbiome of \textit{Populus deltoides}  In: \emph{2012 Department of Energy Genomic Science Awardee Meeting X}, February 26-29, 2012.Poster.

\item \textbf{Shakya M.}, Campbell, J.C., Schadt, C.W.,Podar M. Experimental and Computational Approaches for Microbial Diversity Characterization Using Artificial Communities.In: \emph{2011 Genomic Science Awardee Meeting IX }, April 10-13 , 2011.Poster and Talk.

\item \textbf{Shakya M.}, Campbell, J.C.,Yang,Z.K., Podar M. Experimental and Computational Approaches for Microbial Diversity Characterization Using Artificial Communities In: \emph{International Society of Microbial Ecology 13}, August 22-27, 2010.Poster.

\end{rSection}
%------------------------------------------------
%Teaching experience
%------------------------------------------------

\begin{rSection}{Teaching Experience}{}{}{}
\href{http://www.utk.edu}{\bf University of Tennessee, Knoxville}
\item Teaching Assistant for BCMB 230 : Human Physiology Fall~2009 (2 sections)
\item Teaching Assistant for BIO 140 : Organization and Function of the Cell  Spring~2010 (2 sections)

\href{http://www.swosu.edu}{\textbf{Southwestern Oklahoma State University}}
 \item Teaching Assistant for Biological Sciences Fall~2007-Spring~2008
\item Grader for College Algebra and Trigonometry Spring~2007-Spring~2008

\end{rSection}


%----------------------------------------------------------------------------------------
%	Peer review
%----------------------------------------------------------------------------------------

\begin{rSection}{Peer review activities}

Reviewer for: FEMS Microbiology Ecology, Canadian Journal of Microbiology, Frontiers in Microbiology, Molecular Biology and Evolution, BMC Microbiology, MDPI - \textit{Toxins}
\end{rSection}

%----------------------------------------------------------------------------------------
%	BIOINFORMATIC SKILLS
%----------------------------------------------------------------------------------------


\begin{rSection}{Relevant skills}
{\bf Programming/Scripting}: Python, Perl, R, bioconductor\\
{\bf Bioinformatics}: Genome, Metagenome, Metagenetics, Methylome, SNP analysis, phylogenetics, etc.\\
{\bf Other}: makefile, unix, shell scripting, high performance computing\\
\end{rSection}

%----------------------------------------------------------------------------------------
%	EXAMPLE SECTION
%----------------------------------------------------------------------------------------


%\begin{rSection}{Section Name}

%Section content\ldots

%\end{rSection}

%----------------------------------------------------------------------------------------

\end{document}
